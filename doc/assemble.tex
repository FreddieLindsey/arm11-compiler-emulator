We decided to follow the two-pass assembly method described in the specification. First the ARM assembly file is read in to memory line-by-line. Each line has comments and white-space stripped from it, and if the line is recognised as a label, it is added to the symbol table and ignored. The symbol table is stored using a linked list of symbol structs with methods written to search, add and free an element from this list:

\begin{verbatim}
  typedef struct symbol_table_entry {
    char *name;
    int address;
    struct symbol_table_entry *next;
  } symbol_t;
\end{verbatim}


The second pass then takes each instruction again, and tokenises its arguments. It looks up the first argument (the instruction mnemonic) in a Map-like data structure, which returns a pointer to a function that can deal with that instruction:

\begin{verbatim}
  typedef struct instruction_str {
    char *mnemonic;
    decoded_instruction_t *(*build_instruction)();
    instruction_kt type;
  } instruction_str_t;
\end{verbatim}

The tokenised arguments, as well as other relevant data are passed to this function, which returns a struct holding each part of the instruction as binary. An example for the mov instruction is seen below. It takes an array of strings as arguments, and builds a decoded\textunderscore{}instruction\textunderscore{}structure for this instruction. 

\begin{verbatim}
decoded_instruction_t *build_mov(char **args) {
  decoded_instruction_t *ins = malloc(sizeof(decoded_instruction_t));
  ins->kind = DATA_PROCESS;
  ins->opcode = 13;
  ins->immediate = args[1][0] == '#' || args[1][0] == '=' ? 1 : 0;
  ins->set = 0;
  ins->regn = 0;
  ins->regd = value_to_int(args[0]);
  ins->operand2 = build_operand2(args[1], args[2]);
  return ins;
}
\end{verbatim}

The value\textunderscore{}to\textunderscore{}int function is a utility function that takes a string such as 14 or r2 and returns the integer associated with that string, i.e. 14 and 2 respectively. It also deals with square brackets that could be included in that string. 

~\\

When translating from ARM assembly strings to binary, we were required to implement many different functions to work on strings. We decided to group all of these functions together into a str\textunderscore{}utils library, containing these functions:

\begin{verbatim}
  int is_empty(const char *str);
  char *strduplicate(const char *str);
  int strtoi(char *str);
  char last_char(char *str);
  void trim(char *str);
  void trim_before(char *str);
  void trim_after(char *str);
\end{verbatim} 

The is empty function determines weather a string contains only white-space, strduplicate is a custom implementation of the strdup function, as there were compatibility issues with the built-in version and the trim functions remove trailing/leading white-space from a string

~\\

After the instruction is converted to a decoded instruction structure, it is passed to the encoder, which converts it to the 32-bit binary instruction, there is one function for each type of instruction, shown below is the data transfer function.

\begin{verbatim}
  instruction_t dataprocess_encode(decoded_instruction_t *decoded) {
    const int OFFSET_COND = 0x1C;
    const int OFFSET_I = 0x19;
    const int OFFSET_OPCODE = 0x15;
    const int OFFSET_S = 0x14;
    const int OFFSET_RN = 0x10;
    const int OFFSET_RD = 0xC;
    const int OFFSET_OPERAND2 = 0x0;
    const int COND = 14;
    return (COND << OFFSET_COND) | (decoded->immediate << OFFSET_I) |
      (decoded->opcode << OFFSET_OPCODE) | (decoded->set << OFFSET_S) |
      (decoded->regn << OFFSET_RN) | (decoded->regd << OFFSET_RD) |
      (decoded->operand2 << OFFSET_OPERAND2);
  }
\end{verbatim}

Finally, the binary from this function is written to the output file and is ready for running.