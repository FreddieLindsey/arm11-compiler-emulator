Writing the ARM assembly program was one of the simplest parts of this exercise, as a group we had a good understanding of the language and were able to write a working program to make the LED flash quite quickly. The difficulties were with setting up this program on the raspberry pi. The main challenge faces was the lack of visual feedback given from the pi, the LED either flashed or it didn't. There was no simple way to tell what the problems could be. The emulator helped with this as we were able to test our code on a machine that would give us error messages and feedback from the instructions.

Another challenge was calculating the speed at which the LED should flash. This was more difficult to test on the emulator, as our emulator would not run as fast as a stand alone processor. At some points the LED was flashing so fast it was difficult to tell weather it was flashing at all, or just constantly lit. 

This exercise helped us get an understanding of how to use the ARM assembly language with the raspberry pi's GPIO, it made the extension considerably simpler as we already had an understanding of how to work with the pi.