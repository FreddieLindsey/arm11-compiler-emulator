For the extension we decided to create a simple motor-sport style starting light program to signal the beginning of the race. We connected the raspberry pi up to 5 LEDs to simulate the 5 stages of lighting that a real starting light would show. 
~\\

The raspberry pi was set up by connecting the GPIO pins 7, 8, 9, 10 and 11 to individual LEDs on the breadboard, so that they could be controlled by the ARM program. We chose to use these pins as they are the only 5 consecutive pin numbers on this version of the pi. This allowed LEDs to be controlled in a simple, inuitive fashion, assigning each pin mask to the registers: r7, r8, r9, r10 and r11. 
~\\

The assembly program was written to control the LEDs in 3 different ways. Each of the 3 procedures start the same, by powering the LEDs one by one until all 5 are lit. Additionally, the value stored in the register r3 controls the speed of the countdown. The value stored in register r5 controls what happens after this count-up has finished. In the case it is "0", all 5 of the LEDs will flash at the same time to single the start of the race creating a strobe effect. When it's "1", the LEDs will stay lit which means when the final LED displays the race has started. And when it's "2", the LEDs will perform a ripple effect, each LED will switch on in turn, switching off when the next one turns on - the ripple effect, signalling the start of the race. The effect of the LED can currently only be chosen by setting the register values in the ARM assembly file in the first two assignment lines for registers r5 and r3. These two lines will effectively ensure the various modes and countdown lengths can be easily selected. 
~\\

The pi could easily be connected to more powerful LED clusters, such as the ones used in professional race scenarios, creating an effective starting light system. As these kinds of lights are usually very large, there would be no problems fitting the raspberry pi into the lights housing. If the system was to be used in a smaller setting, for example a toy race track, there may be issues with the raspberry pi being too large.
~\\

The main programming problem was the lack of a test suite to enable easy checks of the codes functionality. We decided not to make one, due to the amount of time needed to build a test suite and the sequential, non interactive nature of the program. In other words, it would not be worth wasting time spent on the extension to create a test suite for this type of program. Still, it meant we had to consider the various cases in turn, recompiling using the assembler and running using the emulator to see if the code worked. Also, the only error messages to help us test the code were segmentation faults, mainly due to the misspelled opcodes and misplaced syntax. This meant errors were hard to spot. Adding further to the complicating nature of debugging was that even if the emulator seemed to verify the codes correctness, there is no concrete guarantee that the pi would ouput the right LEDs.
~\\

In the future it could be possible to attach two switches to act as a selection for the lighting procedure, and possibly another switch to control when to start the count-up. Real starting lights also incorporate a random element into the timings, which we didn't have the time to implement in our program.
